\documentclass[b5paper,opensource]{qyxf-book}

\usepackage{subcaption}
% 添加水印的宏包
\usepackage{draftwatermark}
\SetWatermarkText{钱院学辅}
\SetWatermarkLightness{0.9}
\SetWatermarkScale{0.9}

% 基本不需要改动
\title{大物题解}
\subtitle{Key to Universal Physics}
\author{钱院学辅大物编写小组}
\typo{钱院学辅排版组}
\date{\today}
\version{v1.0}
\sourcepage{\url{https://github.com/qyxf/Tutorials/}}

% 这里可以自定义一些命令
\newcommand{\di}[1]{\mathrm{d}#1}
\newcommand{\p}[2]{\frac{\partial #1}{\partial #2}}
\newcommand{\pp}[2]{\frac{\partial ^2 #1}{\partial #2 ^2}}
\newcommand{\dy}[2]{\frac{\di{#1}}{\di{#2}}}
\newcommand{\ddy}[2]{\frac{\mathrm{d} ^2 #1}{\mathrm{d} #2 ^2}}
\newcommand{\zbj}[4]
{
	\draw (0,0) node[below left] {$ O $};
	\draw [->] (#1,0) -- (#2,0) node[right] {$ x $};
	\draw [->] (0,#3) -- (0,#4) node[right] {$ y $};
}


\begin{document}

%\maketitle 
\tableofcontents

%% 对于章节编写,可以不加封面和目录
\chapter{静电场2}  % 使用章节\chapter{}来做一级标题
\section{选择题}  % 选择题、填空题和解答题使用\section{}

\exercise{1}B  % 题号使用这个命令,会自动生成标记,注记后写主要答案

\solve  % 解答使用这个命令,解答与题号之间空一行
两个平板之间的电场为匀强电场,若记向右为正方向,则电场强度:
\[E=\frac{\sigma}{2\epsilon_0}-\frac{2\sigma}{2\epsilon_0}=-\frac{\sigma}{2\epsilon_0}\]

即电场强度向左,电场线也指向左。由于电势沿着电场线降低,因此$V_B>V_A$。

%% 空一行开始下一个题目
\exercise{2}B

\solve 加入电介质之前,$ C_1,C_2 $电容相同,由于并联,则$ U $也相同,由电容定义$ C=\frac{Q}{U} $,两个电容器所带电荷$ Q_1=Q_2 $。

由于电键断开,因此电容器与外界没有电荷交换,过程中两个电容器的总带电量$ Q=Q_1'+Q_2' $不变。
插入电介质稳定后,相连的极板电势差应为$ 0 $(否则电荷会顺着电势减小的方向移动),则求$ C_1,C_2 $的电势差:
\[\Delta u'=\frac{Q_1'}{C\epsilon_r}=\frac{Q_2'}{C}\]

有:$Q_1'=\epsilon_r Q_2'$,因此$Q_2'<Q_2$,有电荷从$C_2$流向$C_1$,知$C_1$电量增大。

又因为$\Delta u'<\Delta u$,有$E'd<Ed$,$d$不变,可知场强变小,因此选B。

\note 做此类电容题目需要寻找过程中的不变量,若电容器的大小、形状等不变,则过程中电容$ C $不变;若电容器与外界没有电荷交换,则电量$ Q $不变。

\exercise{3}D

\solve A,电位移矢量会受到束缚电荷的影响,书上写道,若电介质是各向异性的,则$ D $与$ E $的方向可不一样。

B,电位移先起始于自由正电荷,终止于自由负电荷,不形成闭合线,不中断。

C,电位移线可以出现在真空中,此时电位移线与电场线相同。

D正确,由电介质中高斯定理可直接得到。

\exercise{4}D

\solve 两板间为匀强电场,电场强度大小为:
\[E=\left|\frac{-\sigma}{2\epsilon_0}-\frac{-2\sigma}{2\epsilon_0}\right|=\frac{\sigma}{2\epsilon_0}\]

则电势差大小:
\[\Delta u=Ed=\frac{\sigma d}{2\epsilon_0}\]

\exercise{5}C

\solve A,放入q2之前球外场强关于O是球对称的,而q2产生的电场关于O不是球对称的,由电场叠加原理,放入q2后两个电场叠加,关于O不是球对称的,因此A错。

B,由静电屏蔽,q2在外球壳内部产生的电场强度被外球壳的电荷所抵消,则导体内部的电场强度仅受到q和内表面上电荷的影响。由于q产生的电场关于O球对称,导体内部的电场强度由于静电平衡,均为0,也是关于O球对称,那么内表面电荷分布也必然是球对称,B错。

C,由于静电平衡,导体内部是等势体,自然内外表面电势差恒为0,C对。

D,$ q_1,q_2 $之间的静电作用力即使库仑力,大小为$ F_c=\frac{1}{4\pi\epsilon_0}\frac{q_1q_2}{r^2} $,显然不为0,D错。

\exercise{6}B

\solve 设均带有$ Q $的电荷,半径为$ R $和$ 2R $。

则孤立导体球的电势:$ U_1=\frac{Q}{4\pi\epsilon_0R} $,$ U_2=\frac{Q}{8\pi\epsilon_0R} $,有$ U_1:U_2=2:1 $。

而电场能量$ W=\frac{QU}{2} $,则可知$ W_1=W_2 $。

\exercise{7}C

\solve 电容与导体带电情况无关,因此电荷面密度$ \sigma $变化前后C不变,而$ Q_1:Q_2=1:2 $。

由$ W=\frac{Q^2}{2C} $,则$ W_1:W_2=1:4 $,电场能量变为原来了的4倍。

\exercise{8}D

\solve 由电介质定义式,$ E=\frac{E_0}{\epsilon_r} $,$ E_0 $为自由电荷的贡献。

而电场强度可拆分为自由电荷与束缚电荷的贡献之和,有$ E=E_0-E_r $。

联立两式,消去$ E_0 $,得到$ E_r=(\epsilon_r-1)E $。

\exercise{9}C

\solve 由高斯定理,P点所在球面仍然只包含$ +q $电荷,因此电场强度不变。事实上,同理可知r-R之间的电场强度也不变。

但对于R之外的电场强度,由高斯定理,大球面外包含的电荷为$ Q+q $,比加入球面之前$ (+q) $大,因此球面外的场强变大。因此对于整个空间上的任意一点,有$ E_2>=E_1 $。

由电势计算式,$ u=\int_a^{+\infty} E\cdot\di{l} $,由定积分的大小性质,有$ u_2>u_1 $,因此电势也增大。

\exercise{10}A

\solve 电容与电容器带电情况无关,因此变化前后电容不变,记为$ C $。

并联之前,充电的电容器电场能量$ W=\frac{Q^2}{2C} $,并联之后,两个电容器电势差相同。

由$ Q=CU $知带电量相同,又因为与外界没有电荷交换,则总带电量相同。则每个电容器电荷量为$ \frac{Q}{2} $。

总电场能量为$ W=2\times \frac{Q^2}{8C}=\frac{Q^2}{4C} $,有$ W_1:W_2=2:1 $,静电能减小。

\section{填空题}
\exercise{11} $4\pi\epsilon_0\epsilon_r\frac{R_1R_2}{R_2-R_1}$

\solve 见教材P149 例6.32,积分时不考虑$E_2$即可。

\exercise{12} $\frac{Q}{4\pi\epsilon_0 R}$ \quad $-\frac{Qq}{4\pi\epsilon_0 R}$

\solve 点电荷在一点产生的电势(见课本)为:
\[u=\frac{q}{4\pi\epsilon_0 R}\]

此处可将半圆形微元$R\di{\alpha}$视为点电荷,在O点产生的电势为:

\[\di{u}=\frac{R\di{\alpha}\lambda_0\sin\alpha}{4\pi\epsilon_0 R}\]

积分可得:
\begin{align*}
u&=\int_{\alpha=0}^{\alpha=\pi} \di{u}\\
&=\int_0^{\pi} \frac{R\di{\alpha}\lambda_0\sin\alpha}{4\pi\epsilon_0 R}\\
&=\frac{\lambda_0}{2\pi\epsilon_0}
\end{align*}

或由于微元到$O$点距离相同,直接用$Q$作为总电量写为$\frac{Q}{4\pi\epsilon_0 R}$即可

当取无穷远处为电势零点时,$ u $表示将单位点电荷移动到无穷远处的过程中,电场力做的功,因此此处需要取负号,并乘$q$即可。

\exercise{13} $\frac{Q}{4\pi\epsilon_0\epsilon}\left(\frac{1}{R_1}+\frac{\epsilon-1}{R_2}\right)$

\solve 当金属球静电平衡时,电荷集中分布在球面上。由高斯定理可求出电场分布:

\begin{equation}
E=\left\{
\begin{aligned}
&\frac{Q}{4\pi\epsilon_0 r^2}\quad &R_2\leqslant r\\
&\frac{Q}{4\pi\epsilon_0\epsilon r^2}\quad &R_1\leqslant r<R_2\\
&0	&r<R_1
\end{aligned}
\right.
\end{equation}

则$ P $点电势为:

\begin{align*}
u&=\int_r^{+\infty}E\cdot\di{r}\\
&=\int_r^{R_1}E\cdot\di{r}+\int_{R_1}^{R_2}E\cdot\di{r}+\int_{R_2}^{+\infty}E\cdot\di{r}\\
&=0+\frac{Q}{4\pi\epsilon_0 R_2}+\frac{Q}{4\pi\epsilon_0\epsilon}\left(\frac{1}{R_1}-\frac{1}{R_2}\right)\\
&=\frac{Q}{4\pi\epsilon_0\epsilon}\left(\frac{1}{R_1}+\frac{\epsilon-1}{R_2}\right)
\end{align*}

\exercise{14} $1:1$

\solve 由电介质中的高斯定理,在$ Q $和$ S $都不变的情况下,$ D $也相同。

\exercise{15} $\frac{q}{4\pi \epsilon_0 r_1^2}$ \quad $\frac{q+Q}{4\pi\epsilon_0 r_2^2}$ \quad $\frac{1}{4\pi\epsilon_0}\left(\frac{q}{r_1}+\frac{Q}{R_2}-\frac{q+Q}{r_2}\right)$

\solve 由高斯定理计算得到电场强度:
\begin{equation}
E=\left\{
\begin{aligned}
&\frac{Q}{4\pi\epsilon_0 r^2}\quad &R_2\leqslant r\\
&\frac{q+Q}{4\pi\epsilon_0 r^2}\quad &R_1\leqslant r<R_2\\
&0	&r<R_1
\end{aligned}
\right.
\end{equation}

将A,B离O点的距离$ r_1,r_2 $代入即可得到场强大小。

而电势差则需要积分,积分路径为:A-OA连线与外球壳交点-沿外球壳移动到OB与外球壳交点-沿OB到B。球壳为等势面,因此在球壳上移动电场力做功为0。则电势差:
\begin{align*}
\Delta u&=\int_{r_1}^{R_2}E\cdot\di{r}+0+\int_{R_2}^{r_2}E\cdot\di{r}\\
&=\frac{q}{4\pi\epsilon_0}\left(\frac{1}{r_1}-\frac{1}{R_2}\right)+\frac{q+Q}{4\pi\epsilon_0}\left(\frac{1}{R_2}-\frac{1}{r_2}\right)
&=\frac{1}{4\pi\epsilon_0}\left(\frac{q}{r_1}+\frac{Q}{R_2}-\frac{q+Q}{r_2}\right)
\end{align*}

\exercise{16} 尖端放电

\exercise{17} $ -\frac{1}{12\pi\epsilon_0 R(3Q+q)} $

\solve 静电平衡时,金属球的电荷全部分布在表面上,且整个金属球为一个等势体,因此求出金属球上任意一点的电势即为金属球的电势,这里我们来求O点电势。

由电势叠加原理,O点电势为球表面电荷和-q产生的电势之和。球表面电荷到O点的距离都相同。

因此所有球面上的电荷微元$\di{q}$在O点产生的电势均为$\frac{\di{q}}{4\pi\epsilon_0 R}$,因此这部分的电势为$\frac{-Q}{4\pi\epsilon_0 R}$。

而$ -q $在O点产生的电势为$\frac{-q}{4\pi\epsilon_0 \times 3R}$,两部分相加即得到答案。

\exercise{18} $\frac{2\pi l\epsilon_0\epsilon_r}{\ln \frac{R_2}{R_1}}$

\solve 设内外面带电$ +Q $和$ -Q $,$ l>>R_1,R_2 $,因此可以当作无限长圆柱面。

由书P141例6.27,无电介质时电容为$ C=\frac{2\pi l\epsilon_0}{\ln \frac{R_2}{R_1}} $。

插入电介质后电容变为原来的$ \epsilon_r $倍,因此得到答案。

\exercise{19} 击穿场强

\exercise{20} $ 0 $\quad$ -q_1 $

\solve 外球壳静电平衡,因此为了保证外球壳内部场强为0,由高斯定理,外球壳内表面必须带电荷$ -q1 $。

当我们先不考虑外球壳外表面的电荷(如果有的话)时,外球壳外表面所处位置电场强度处处为0。

由于接地,外球壳外表面和大地电势为0,因此外表面不能带电荷(否则会在外表面和大地的导线处产生电场强度,导致外球壳外表面和大地不为等势体)。

\section{计算题}

\exercise{21}

\solve 

(1).

记金属片左侧面与左边极板间距为$ a $,右侧面与右边极板间距为$ b $,则有$ d=a+t+b $。

可将电容器视作两个电容的串联:左边极板和金属片左侧(记为1),右边极板和金属片右侧(记为2)。则:
\[C_1=\frac{\epsilon_0 S}{a}.C_2=\frac{\epsilon_0 S}{b}\]

串联之后,总电容:
\[C=\frac{C_1C_2}{C_1+C_2}=\frac{\epsilon_0 S}{a+b}=\frac{\epsilon_0 S}{d-t}\]

(2).

由(1)推导知,$ C $与$ a,b $无关,因此放置位置对电容值无影响。

\exercise{22}

\solve 
(1).

$ \because A$接地,$ \therefore u_A=0 $

由无限远处为电势零点:

\begin{equation}
\therefore u_A=\int_a^{3a} E_1\cdot\di{r} +\int_{3a}^{+\infty} E_2\cdot\di{r} =0
\end{equation}

记$ A $带点$ q_1 $,则有
\begin{equation}
\begin{aligned}
E_1&=\frac{q_1}{4\pi r^2\epsilon_0\epsilon_r}=\frac{q_1}{8\pi r^2\epsilon_0}\\
E_2&=\frac{q_1+q_2}{4\pi r^2\epsilon_0}\\
\therefore u_A&=\int_a^{3a}\frac{q_1}{8\pi r^2\epsilon_0}\di{r}+\int_{3a}^{+\infty}\frac{q_1+q_2}{4\pi r^2\epsilon_0}\di{r}\\
&=\frac{q_1}{8\pi\epsilon_0}\left(\frac{1}{a}-\frac{1}{3a}\right)+\frac{q_1+q_2}{4\pi\epsilon_0}\times \frac{1}{3a}\\
&=\frac{q_1}{12\pi\epsilon_0a}+\frac{q_1+q_2}{12\pi\epsilon_0a}=0
\end{aligned}
\end{equation}

解得:$ q_1=-\frac{q_2}{2} $

(2).

\[u_B=\int_{3a}^{+\infty}E_2\cdot \di{r}=\frac{q_1+q_2}{12\pi\epsilon_0a}=\frac{q_2}{24\pi\epsilon_0a}\]

(3).

电容大小与电容器实际带电情况无关,因此可以考虑内外球分别带$ +Q/-Q $电荷时,计算出电容,即为答案。此时有:
\[\Delta u=\int_a^{3a}E\cdot \di{r}=\int_a^{3a}\frac{Q}{4\pi r^2\epsilon_0\epsilon_r}\di{r}=\frac{Q}{24\pi\epsilon_0a}\]
\[\therefore C=\frac{Q}{\Delta u}=24\pi\epsilon_0a\]

(4).
\begin{align*}
W&=\int_a^{3a}\frac{1}{2}\epsilon_0 E_1^2\di{V}+\int_{3a}^{+\infty}\frac{1}{2}\epsilon_0 E_2^2\di{V}\\
&=\int_a^{3a}\frac{q_2^2}{128\pi r^2\epsilon_0} \di{V}+\int_{3a}^{+\infty}\frac{q_2^2}{32\pi r^2\epsilon_0}\di{V}\\
&=\frac{q_2^2}{64\pi\epsilon_0a}
\end{align*}

\exercise{23}

\solve 取圆柱高斯面,由高斯定理可求出电场强度分布:
\begin{equation}
E=\left\{
\begin{aligned}
&0\quad &R_2<r\\
&\frac{\lambda}{2\pi\epsilon_0 r}\quad &R_1<r\leqslant R_2\\
&0	&r\leqslant R_1
\end{aligned}
\right.
\end{equation}

则电势差可以积分得到:

\begin{align*}
u_{R_1}-u_{R_2}&=\int_{R_1}^{R_2} \frac{\lambda}{2\pi r\epsilon_0} \di{r}\\
&=\frac{\lambda}{\pi\epsilon_0}\ln\frac{R_2}{R_1}\\
&=500
\end{align*}


解得:
\[\lambda=\frac{2\pi\epsilon_0\times 500}{\ln\frac{R_2}{R_1}}=1.208\times 10^{-8}\]

\exercise{24}

\solve 
(1).记金属板靠近$ \sigma_1 $一侧的电荷密度为$ \sigma_1' $,靠近$ \sigma_2 $一侧的电荷密度为$ \sigma_2' $。

由静电平衡,则金属板内部一点的场强为$ 0 $,则有:
\[E=\frac{\sigma_1}{2\epsilon_0}+\frac{\sigma_1'}{2\epsilon_0}-\frac{\sigma_2'}{2\epsilon_0}-\frac{\sigma_2}{2\epsilon_0}=0\]

又因为金属板不带电,有:
\[\sigma_1'+\sigma_2'=0\]

解得:$\sigma_1'=-10^{-8}C/m^2,\sigma_2'=10^{-8}C/m^2$

(2).原电场强度:
\[E=\frac{\sigma_1}{2\epsilon_0}-\frac{\sigma_2}{2\epsilon_0}=10^{-8}\frac{1}{\epsilon_0}\]

则原电势差:
\[\Delta u=Ed=10^{-10}\frac{1}{\epsilon_0}\]

插入金属板后,由于金属板左右两侧带电相反,因此对两边的空间场强不产生影响,此结论证明详见课本P126上方。

因此两板间空间处电场强度不变,$ E'=10^{-8}\frac{1}{\epsilon_0} $,金属板内部电场强度为$ 0 $。

因此电势差为:
\[\Delta u'=E'd=10^{-8}\frac{1}{\epsilon_0}\times(0.01-0.002)=8\times 10^{-11}\frac{1}{\epsilon_0}\]

电势差的变化:
\[U=2\times10^{-11}\frac{1}{\epsilon_0}=2.26(V)\]

\end{document}
