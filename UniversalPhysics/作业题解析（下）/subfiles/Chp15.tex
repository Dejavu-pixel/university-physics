\chapter{波动光学1(干涉)}
\makeatletter
\@addtoreset{equation}{exercises}
\makeatother
\section{选择题}
\exercise A

\solve 相干长度(书上P129):一切实际光源发射的光是一个个的波列,当两个分光束的光程差大于波列长度L时,将不能发生干涉现象。这个波列长度L成为该光源的相干长度。 

\exercise C

\solve 两束光不能观察到相干现象的本质是两束光不是相干光。相干光要求:频率相同,光矢量振动方向平行且具有恒定的相位差。

\exercise A

\solve 光源的空间相干性指的是光源的相干长度,由光源的线度决定。可以理解为光源本身的尺寸影响了光源发出的光的波列的长度。

\exercise C

\solve 由于一个狭缝的宽度变窄,说明其中一束光的光强变弱,但是其振动的频率和相位没有发生变化,故两束光仍然是相干光,仍然会发生干涉现象。由双缝干涉间距公式$\Delta=\frac{D\lambda}{d}$,可知条纹的间距不变。由于此时两束光的最大光强不同,故原来光强为零的地方现在不再为零。

\exercise A

\solve 振幅相等说明光强相等,则由此产生的光的干涉现象中,最小光强为零,最大光强为单束光光强的两倍。

\exercise A

\solve 本题考查介质中光程差的计算方法。由于加入玻璃片后,光在玻璃片中的光程差发生改变,增加了(n-1)d,即增加了5λ的光程差,故不影响原有的明纹暗纹分布,中央明纹处依旧是明纹。

\exercise D

\solve 解析:由于要研究透射光的相干现象。做出光路图(光路图一),由图可知两束光的光程差为2nd,由相长干涉的条件(公式四)可知最小厚度即为λ/(2n)。

\exercise B

\solve 解析:研究光的干涉现象的关键是要分析清楚光程差的变化情况。设凸透镜离开平玻璃的距离为l,由牛顿环的中心光程差(公式五),可知现在中心点的光程差为(公式六),随着l的增加光程差也在增加,故中心点的光程差将依次满足明纹暗纹的条件,故中心点会出现明纹暗纹交替的现象。对于第k级暗纹,原来的rk为(公式七),现在为(公式八),可知(公式九),故条纹向中心收缩。

\exercise C

\solve 注意审题,是从下向上观察,即观察透射光的干涉现象(类比第七题)。由于半波损失发生在光从光疏介质射向光密介质,在界面上发生反射时,由于玻璃的折射率为1.5,故两束光都没有半波损失,由出现暗纹的条件(公式十)可知最小距离为78.1nm。

\exercise B

\solve 扩展光源发出的波矢量方向极多,相干性很差,如白炽灯。但是同一个角度发出的光在屏幕上在同一个圆周上,可类比点电荷的电场线分布考虑。
\section{填空题}
\exercise $5\times10^{-7}\mathrm{m}$ 

\solve 由于光的相位的该变量为3π,那么光程差即为1.5λ,由频率可以求出波长,就可以求出介质薄片的厚度,具体如图:(图一)

\exercise 563.6

\solve 本题考查迈克尔逊干涉仪中的条纹距离公式,即δd=Nλ/2,代入相关数据即可解得。

\exercise $3d$

\solve  又是透射光的光程差问题,与第七题和第九题一样,此处不再赘述。

\exercise 两个次波源 \quad 相干波源 

\solve 书上介绍了两种获得相干光源的办法:分波阵面法和分振幅法。前者是在光源发出的同一波列的波面上取出两个次波源作为相干波源,如杨氏干涉。后者是把同一波列的波分为两束光波,如薄膜干涉。

\exercise 9λ/4n2

\solve (2是n的下角标) 做出光路图如图所示,由于两束光均为从光疏介质射向光密介质并且在界面分界出发生反射,故两束光均存在半波损失,故光程差没有半波损失。由干涉出现暗纹的条件可知厚度。(光路图二)

\exercise $(r1/r2)^2$

\solve 分别写出r1和r2的表达式,则可知液体的折射率。(公式十一)

\exercise $4*10^-5$

\solve 由劈尖干涉的公式(公式十二)可知sinθ的值,根据几何关系和小角度近似(公式十三)可知小纸条的厚度。

\exercise $xd/5D$

\solve 直接由杨氏双缝干涉中相邻的明纹或者暗纹之间的距离公式(公式十四),且第零级明纹和第五级明纹之间的距离为五个δx,故可知(公式十五)

\exercise 

\solve 对于一般位置分析,做出光路图(光路图三)可知光程差的表达式为(公式十六),在中心处由于d=0,故中心处的光程差为(公式十七),满足干涉出现暗纹的条件。

\exercise 

\solve 考查迈克尔逊干涉仪的相关概念。当M1撇和M2平行时,可观察到同心圆的条纹分布,这是等倾干涉。当M1撇和M2不平行时,由于空气隙的存在,会观察到类似于等厚干涉的等距直线条纹。

\section{解答题}
\exercise 

\analysis
本题考查劈尖干涉,只要能正确理解公式的含义就可以顺利解出,难度不大。

\solve
设劈尖的顶角为$\theta$,相邻暗纹距离为$a$,薄膜厚度为$d$,AB长度为$l$,则:
\begin{gather*}
	a\sin\theta=\dfrac{\lambda}{2n_1}\\
	l=10a
\end{gather*}
联立得:
\[
d=l\sin\theta=\dfrac{5\lambda}{n_1}
\]
代入数据得:
\[
d=1.431\times 10^{-6}\mathrm{ m}
\]

\exercise 

\analysis
本题考查劈尖干涉的光程差的分析,由于条纹整体左移,说明光程差变大了,即左边的空气隙的厚度变大了,说明工件表面有凹陷,且可以定量算出凹陷的深度。

\solve
条纹向左移动说明左侧空气隙厚度变大,即左侧发生凹陷。

设凹陷的深度为$\Delta d$,则:

原来:$2nd_1+\frac{\lambda}{2}=(2k+1)\frac{\lambda}{2}$

现在:$2n(d_1+\Delta d)+\frac{\lambda}{2}=[2(k+1)+1]\frac{\lambda}{2}$

作差得:$\Delta d=\frac{\lambda}{2}$

\exercise 

\analysis
本题考查薄膜干涉的两种情况。如果观察反射光可以直接利用薄膜干涉的公式(公式十八),根据k的取值判断光的颜色。如果观察透射光,可以利用作业第七题,第九题提到的算法同样判断k的值得到光的颜色。

\solve
(1) 正面观察
\[
\delta=2nd+\dfrac{\lambda}{2}=2k\cdot \dfrac{\lambda}{2}
\]
即
\[
\lambda=\dfrac{4nd}{2k-1}(k=1,2,3...)
\]
其中满足400nm$\leqslant\lambda\leqslant$760nm\footnote{尽管据报道,波长为390nm也是可见的,但考试用本题解所述范围也正确。}的解为668.8nm($k=2$)和401.28nm($k=3$)。查表知二者分别为红色、紫色。
%有效数字?

(2) 背面观察
\[
\delta=2nd=2k\cdot \frac{\lambda}{2}
\]
即
\[
\lambda=\dfrac{2nd}{k}(k=1,2,3...)
\]
其中满足400nm$\leqslant\lambda\leqslant$760nm的解为501.6nm($k=2$)。查表知其颜色为绿色。

\exercise 

\analysis
本题要求类比牛顿环的分析方法分析干涉现象,关键是要抓住光程差的分析,前两问非常基础。第三问关于疏密的判断可以类比第八题,即判断δr,由于δr是关于θ的单元函数,故可以由θ的变化判断出疏密的变化。
\solve

