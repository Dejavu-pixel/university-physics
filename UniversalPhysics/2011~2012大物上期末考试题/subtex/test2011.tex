\chapter{2011年期末试题}
\section{选择题}
\exercise{1}
一枚在星际空间飞行的火箭,当它以恒定的速率燃烧燃料时,运动学方程为$x=ut+u(\dfrac{1}{b}-t)\ln(1-bt)$。其中$u$是喷出气流相对于火箭的喷射速度,是一个常量,$b$是与燃烧速率成正比的一个常量,则此火箭的速度与加速度的表达式分别为
\optionss{-u\ln(1-bt),\dfrac{-bu}{1-bt}}
{-u\ln(1-bt),\dfrac{bu}{1-bt}}
{u\ln(1-bt),\dfrac{-bu}{1-bt}}
{u\ln(1-bt),\dfrac{bu}{1-bt}}

\exercise{2}
一光滑的内表面半径为10cm的半球形碗。如图所示,以角速度$\omega$绕其对称轴OC旋转,已知放在碗内的一个小球P相对于碗静止,其位置高于碗底4cm,由此可推知碗的旋转角速度约为
\option{13\textrm{rad/s}}
{17\textrm{rad/s}}
{10\textrm{rad/s}}
{18\textrm{rad/s}}

\begin{figure}[!h]
	\centering
	\subfigure[选择2 示意图]{
		\begin{minipage}[t]{0.4\linewidth}
			\includegraphics[width=\textwidth]{./illus/2011_1.ai}
	\end{minipage}}
	\quad
	\subfigure[选择4 示意图]{
		\begin{minipage}[t]{0.4\linewidth}
			\includegraphics[width=\textwidth]{./illus/2011_2.ai}
	\end{minipage}}
	\caption{两张题图}
\end{figure}

\exercise{3}已知地球的质量为$m$,太阳的质量为$M$,地心与日心的距离为$R$,引力常量为$G$,则地球绕太阳做圆周运动的角动量为
\option{m\sqrt{GMR}}
{\sqrt{\dfrac{GMm}{R}}}
{m\sqrt{\dfrac{GM}{R}}}
{\sqrt{GMmR}}

\exercise{4}
一轻绳跨过一具有水平光滑轴,质量为$M$的定滑轮,绳的两端分别悬有质量为$m_1,m_2$的物体($m_1<m_2$),如图所示,绳与轮之间无相对滑动,分别考虑轮为实心和空心的情况,当$m_2$下降相同高度后获得的速率分别为$v_1$和$v_2$,试确定两者大小关系
\option{v_1=v_2}
{v_1<v_2}
{v_1>v_2}
{\text{无法确定}}

\exercise{5}
根据相对论力学,动能为0.25MeV的电子,其运动速度约等于(电子的静能为$m_0c^2$=0.5MeV,$c$为真空中光的速度)
\option{0.1c}
{0.5c}
{0.75c}
{0.85c}

\exercise{6}两个同心均匀带电球面,半径分别为$R_1,R_2(R_1<R_2)$所带电量分别为$Q_1,Q_2$,设某点与球心的距离为$r$,当$R_1<r<R_2$时,该点的电场强度的大小为
\optionss{\dfrac{1}{4\pi\varepsilon_0}\dfrac{Q_1+Q_2}{r^2}}
{\dfrac{1}{4\pi\varepsilon_0}\dfrac{Q_1-Q_2}{r^2}}
{\dfrac{1}{4\pi\varepsilon_0}(\dfrac{Q_1}{r^2}+\dfrac{Q_2}{R^2})}
{\dfrac{1}{4\pi\varepsilon_0}\dfrac{Q_1}{r^2}}

\exercise{7}
有一沿水平方向放置的带电直线,长为L,电荷线密度为$\lambda$,则带电直线右侧延长线上距离带电直线左端点为$r(r>L)$处的电势大小为
\optionss{\dfrac{\lambda}{4\pi\varepsilon_0}\ln\dfrac{r+L}{r}}
{\dfrac{\lambda}{4\pi\varepsilon_0}\ln\dfrac{r-L}{r}}
{\dfrac{\lambda}{4\pi\varepsilon_0}\ln\dfrac{L}{r+L}}
{\dfrac{\lambda}{4\pi\varepsilon_0}\ln\dfrac{L}{r-L}}

\exercise{8}
将一空气平行板电容器接到电源上,充电到一定电压后,在保持与电源连接的情况下,再将一块与平板面积相同的金属板平行的插入两极板间,金属板的插入及所处位置不同,对电容器储存电能的影响为:
\options    
{储能减少,但与金属板的位置无关}
{储能减少,且与金属板的位置有关}
{储能增加,但与金属板的位置无关}    
{储能增加,且与金属板的位置有关}

\begin{figure}[!h]
	\centering
	\subfigure[选择8 示意图]{
		\begin{minipage}[t]{0.3\linewidth}
			\includegraphics[width=\textwidth]{./illus/2011_3.ai}
	\end{minipage}}
	\quad
	\subfigure[选择9 示意图]{
		\begin{minipage}[t]{0.3\linewidth}
			\includegraphics[width=\textwidth]{./illus/2011_4.ai}
	\end{minipage}}
	\quad
	\subfigure[选择10 示意图]{
		\begin{minipage}[t]{0.3\linewidth}
			\includegraphics[width=\textwidth]{./illus/2011_5.ai}
	\end{minipage}}
	\caption{三张题图}
\end{figure}

\exercise{9}
如图所示,一长直导线中部弯成半径为$r$的半圆形,导线中通以恒定电流$I_1$,则弧心O点处的磁感应强度的大小和方向分别是
\optionss{\dfrac{\mu_0I}{2\pi r}+\dfrac{\mu_0I}{4r}\text{,向外}}
{\dfrac{\mu_0I}{2\pi r}+\dfrac{\mu_0I}{4r}\text{,向里}}
{\dfrac{\mu_0I}{4r}\text{,向外}}
{\dfrac{\mu_0I}{4r}\text{,向里}}

\exercise{10}
在圆柱形空间内有一磁感应强度为$B$的均匀磁场垂直于纸面向里,$B$的大小以恒定速率变化,有一长度为$L$的金属棒先后放在磁场的不同位置,位置1$(a,b)$感应电动势大小为$\varepsilon_1$,位置2$(a',b')$感应电动势大小为$\varepsilon_2$,如图所示,则
\optionss{\varepsilon_1=\varepsilon_2\neq0}
{\varepsilon_1<\varepsilon_2}
{\varepsilon_1>\varepsilon_2}
{\varepsilon_1=\varepsilon_2=0}

\section{填空题}

\exercise{1}
质点沿半径为$R$的圆做圆周运动,某一时刻其加速度大小为$a$,方向与位矢的夹角为$\theta$,则该时刻质点的速率为\ul,切向加速度的大小为\ul。

\exercise{2}
质量为$m$=1kg的质点,从静止出发,在水平面内沿x轴正向运动。其所受合力方向与运动方向相同,合力大小为$F=3+2x$,物体在开始运动的3m内合力做功$A=$\ul;$x=3$时,其速率$v=$\ul。

\exercise{3}
长为$L$,质量为$M$的均匀细杆,以及一长为$L$,质量为$M$的单摆(绳的质量忽略不计),今用同样的弹丸(质量均为$m$)以同样的速度$v$沿水平方向分别击中杆和单摆的下端,并与之合为一体,则击中后的瞬间杆的角速度为\ul,单摆的角速度为\ul。

\exercise{4}
如图所示,图中实线为某电场的电场线,虚线为等势面,则$E_A,E_B,E_C$的大小关系为\ul,$U_A,U_B,U_C$的大小关系为\ul。


\begin{figure}[!h]
	\centering
	\subfigure[填空4 示意图]{
		\begin{minipage}[t]{0.4\linewidth}
			\includegraphics[width=\textwidth]{./illus/2011_6.ai}
	\end{minipage}}
	\quad
	\subfigure[填空5 示意图]{
		\begin{minipage}[t]{0.4\linewidth}
			\includegraphics[width=\textwidth]{./illus/2011_7.ai}
	\end{minipage}}
	\caption{两张题图}
\end{figure}

\exercise{5}
A、B为真空中两个平行的无限大均匀带电平面,平面间的电场强度大小为$E_0$,$B$平面外侧的电场强度为$\dfrac{E_0}{3}$,方向由A指向B,则A、B平面的电荷密度分别为$\sigma_1=$\ul,$\sigma_2=$\ul。

\exercise{6}
一个通有电流$I$的导体,厚度为$D$,横截面积为$S$,放在磁感应强度为$B$的匀强磁场(磁感应强度为 $B$)中,磁场方向垂直于导体的侧平面,现测得导体上下两面电势差为$V$,此导体的霍尔系数为\ul。

\exercise{7}
如图所示,在无限长直载流导线的右侧有面积为$S_1,S_2$的两个矩形回路,两个回路和长直载流导线在同一平面内,且电流方向和矩形回路的一边平行,则通过面积为$S_1$的矩形回路的磁通量和通过面积为$S_2$的矩形回路的磁通量之比为\ul。

\begin{figure}[!h]
	\centering
	\subfigure[填空7 示意图]{
		\begin{minipage}[t]{0.4\linewidth}
			\includegraphics[width=\textwidth]{./illus/2011_8.ai}
	\end{minipage}}
	\quad
	\subfigure[填空8 示意图]{
		\begin{minipage}[t]{0.4\linewidth}
			\includegraphics[width=\textwidth]{./illus/2011_9.ai}
	\end{minipage}}
	\caption{两张题图}
\end{figure}

\exercise{8}
一无铁芯的长直密绕螺线管,在保持半径和总匝数不变的情况下,把螺线管稍微拉长一点,不考虑漏磁的理想情况下,则它的自感系数将\ul(变大,变小或不变)。


\section{解答题}%用\vspace控制答题空间
%\begin{wrapfigure}{1}[r][1em]\end{wrapfigure}
\exercise{1}
如图所示,一个转动惯量为$J$,半径为$R$的定滑轮上面绕有细绳,并沿水平方向拉着一个质量为$M$的物体A,整个装置静止且细绳处于拉直状态。现有一质量为$m$的子弹在距转轴$\frac{R}{2}$处。试求:

\exercisequestion{1}求子弹射入并停留在滑轮边缘后,滑轮开始转动的角速度 。

\exercisequestion{2}如果定滑轮拖着A刚好转动一周停止,求A与地面的摩擦系数。(轴上摩擦力忽略不计);

\begin{figure}
	\begin{flushright}
		\includegraphics[width=0.4\textwidth]{./illus/2011_10.ai}
		\caption{解答1 示意图}
	\end{flushright}
\end{figure}

\exercise{2}
在6000m的高空大气层产生了一个$\pi$介子,以速度$v=0.998c$飞向地球,假定该$\pi$介子在其自身的静止系中的寿命约等于其平均寿命$2\times 10^{-6}$,试从下面两个角度,即地球上的观察者和介子静止系中观察者,来判断该介子能否到达地球。

\exercise{3}
一半径为$R$的无限长带电圆柱,其电荷体密度$\rho=\rho_0r(r<R)$,$\rho_0$为常量,求电场强度分布。

\exercise{4}
半径为$R$的无限长带电圆柱导体,通有电流$I$,$I$均匀分布在其横截面上。

\exercisequestion{1}试求外的磁感应强度$B$的分布。

\exercisequestion{2}在柱体内挖一个空心圆柱,空心部分的半径为$b$,轴线与圆柱轴线平行但不重合,两者相距为$a$。若此时圆柱体内电流为$I$,均匀分布在其横截表面上,试求圆柱轴线和空心圆柱轴线上的磁感应强度$B$的大小。

\exercise{5}
有一根辐条的轮子在均匀磁场中转动,转动轴与磁感应强度$B$平行,如图所示,轮子和辐条都是导体,辐条长为$R$,轮子每秒转$N$圈。两条导线a和b通过各自的电刷分别和轮轴和轮缘接触。

\exercisequestion{1}试求a,b间的感应电动势$\varepsilon_1$;

\exercisequestion{2}若在a,b间接一个电阻使辐条中的电流为$I$,试问$I$的方向如何?

\exercisequestion{3}试求这时磁场作用在辐条上的力对轮轴的力矩$M$的大小。
\begin{figure}[!h]
	\begin{flushright}
		\includegraphics[width=0.4\textwidth]{./illus/2011_11.ai}
		\caption{解答5 示意图}
	\end{flushright}
\end{figure}

\newpage
\section{参考答案}
\subsection{选择题和填空题}
\choosing{1}{10} BAACC DDCCC

\filling{\sqrt{aR|\cos\theta|}\quad a\sin\theta}
{18\textrm{J}\quad6\textrm{m/s}}
{\dfrac{mv}{(m+\frac{1}{3}M)L}\quad\dfrac{mv}{(m+M)L}}
{E_A<E_B<E_C\quad U_A>U_B>U_C}
{\dfrac{4}{3}\varepsilon_0E_0\quad -\dfrac{2}{3}\varepsilon_0E_0}
{\dfrac{VD}{IB}}
{1(1:1)}
{\textrm{变小}}

部分题目解析:

\exerciseanswer{选择}{4}

\tips 两种情况的区别在于,实心轮质量分布靠近轮心,转动惯量较小,加速度大。

\exerciseanswer{选择}{9}

\tips 由毕奥—萨法尔定律,直径部分在O点产生的磁感应强度为零。不可与无线长直导线模型混淆,认为磁感应强度是无穷。

\exerciseanswer{选择}{10}

\solve 均匀变化磁场产生的感应电动势有两种求解方法:

(1) 感生电场叠加。这是普适的方法,往往需要投影、积分,可能较麻烦。常用于计算一段导体上的电动势。在本题中,设导线中心到O的距离为$d$,则:
\begin{align*}
\varepsilon&=\int_{-\frac{1}{2}L}^{\frac{1}{2}L}\left|\dy{B}{t}\right|\cdot\dfrac{\sqrt{x^2+d^2}}{2}\cdot\dfrac{d}{\sqrt{x^2+d^2}}\di{x}\\
&=\left|\dy{B}{t}\right|\cdot\dfrac{Ld}{2}
\end{align*}
可见$\varepsilon$与$d$成正比,故选C。

(2) 法拉第电磁感应定律法。此法可计算环路、一段导体的电动势,只要面积容易计算。感生电场方向垂直于相应半径,故任意场点与O的连线上电动势为零,则一段导体的电动势等于相应环路的电动势。本题中:
\begin{align*}
\left|\varepsilon\right|&=\left|\dy{\phi}{t}\right|\\
&=\left|\dy{B}{t}\right|S\\
&=\left|\dy{B}{t}\right|\cdot\dfrac{Ld}{2}
\end{align*}
这样相对更容易计算。

\exerciseanswer{填空}{5}
由无限带电平面场强公式$E=\dfrac{\delta}{2\varepsilon_0}$得(向右为正向):
\begin{align*}
\dfrac{\delta_1}{2\varepsilon_0}-\dfrac{\delta_2}{2\varepsilon_0}&=E_0\text{(平面间)}\\
\dfrac{\delta_1}{2\varepsilon_0}+\dfrac{\delta_2}{2\varepsilon_0}&=\dfrac{E_0}{3}\text{(B
	平面右侧)}
\end{align*}
解得:
\begin{gather*}
\delta_1=\dfrac{4}{3}\varepsilon_0E_0\\
\delta_2=-\dfrac{2}{3}\varepsilon_0E_0
\end{gather*}


\exerciseanswer{填空}{8}%查书页码、版本、具体叙述

\solve
参照大学物理课本(黑皮)第……页的推导,计算公式为$L_0=\mu_0n^2V=\mu_0S\dfrac{N^2}{l}$,其中$n$是单位长度上的匝数,$N$是总匝数,$S$是横截面积,$l$是螺线管长度,按此分析,答案应是减小。该模型针对的是无限长螺线管,且认为半径很小,管内磁感应强度处处相等,这里并不完全严谨。(我回去再看看书)

感兴趣的同学可以参考相关文献,如邰爱东给出了两种准确的计算方法\footnote{邰爱东.有限长直密绕螺线管的自感系数[J].物理与工程,2003(06):8-9.}。


\subsection{解答题}
\solves{1}%普通格式,英文括号,然后空一下
(1) 射入时,由角动量守恒:
\begin{gather*}
R\cdot mv_0\sin\dfrac{5\pi}{6}=J\omega+mR^2\omega+MR^2\omega\\
\omega=\dfrac{mv_0R}{2(J+mR^2+MR^2)}
\end{gather*}
(2) 由动能定理:
\begin{gather*}
-\dfrac{1}{2}(J+mR^2+MR^2)\omega^2=-\mu Mg\cdot 2\pi R\\
\mu=\dfrac{m^2v_0^2R}{16\pi Mg(J+mR^2+MR^2)}
\end{gather*}

\solves{2}
对地球上的观察者:
\begin{gather*}
\tau=\dfrac{\tau_0}{\sqrt{1-{\left(\frac{v}{c}\right)}^2}}=\dfrac{2\times 10^{-6}}{\sqrt{1-0.998^2}}=3.16\times 10^{-5}(\textrm{s})\\
t=\dfrac{s_0}{v}=\dfrac{6000}{0.998c}=2.00\times 10^{-5}(\textrm{s})<\tau
\end{gather*}
故能到达地球。

对$\pi$介子系观察者:
\begin{align*}
s&=s_0\sqrt{1-{\left(\frac{v}{c}\right)}^2}\\
t'&=\dfrac{s}{v}=\dfrac{s_0}{v}\sqrt{1-{\left(\frac{v}{c}\right)}^2}\\
&=2.00\times 10^{-5}\times 0.0632\\
&=1.27\times 10^{-6}(\textrm{s})<\tau_0
\end{align*}
故能到达地球。

\solves{3}
取半径为$r$、高为$h$的高斯面,由高斯定理:

$r<R$时,
\begin{gather*}
E\cdot 2\pi rh=\dfrac{1}{\varepsilon_0}\int_0^r \rho\cdot 2\pi rh\di{r}=\dfrac{2\pi h\rho_0}{\varepsilon_0}\int_0^rr^2\di{r}\\
E=\dfrac{\rho_0r^2}{3\varepsilon_0}
\end{gather*}
$r>R$时,
\begin{gather*}
E\cdot 2\pi rh=\dfrac{1}{\varepsilon_0}\int_0^R \rho\cdot 2\pi rh\di{r}=\dfrac{2\pi h\rho_0}{\varepsilon_0}\int_0^Rr^2\di{r}\\
E=\dfrac{\rho_0R^3}{3\varepsilon_0r}
\end{gather*}
\therefore$E=$
$\begin{cases}
\vspace{0.3em}%不会调cases的行距。。
\dfrac{\rho_0r^2}{3\varepsilon_0},&r<R\\
\vspace{0.2em}
\dfrac{\rho_0R^3}{3\varepsilon_0r},&r>R
\end{cases}$
,方向沿场点处高斯面的法向量向外。

\note 这种求分布的题最好还是写上方向。

\solves{4}
(1) 在距离导体中心$r$处,取一半径为$r$的环路,由安培环路定理:

$r<R$时,
\begin{gather*}
B\cdot 2\pi r=\mu_0\dfrac{I}{\pi R^2}\cdot\pi r^2\\
B=\dfrac{\mu_0Ir}{2\pi R^2}
\end{gather*}
$r>R$时,
\begin{gather*}
B\cdot 2\pi r=\mu_0I\\
B=\dfrac{\mu_0I}{2\pi r}
\end{gather*}
\therefore$B=
\begin{cases}
\vspace{0.3em}
\dfrac{\mu_0Ir}{2\pi R^2},&r<R\\
\vspace{0.2em}
\dfrac{\mu_0I}{2\pi r},&r>R
\end{cases}$

方向沿该点处环路的切向量,与电流方向符合右手定则。

(2) 此时,电流面密度$\sigma=\dfrac{I}{\pi R^2-\pi b^2}$。%求调行距。。

该系统可看做半径为$R$、电流面密度为$\sigma$的圆柱和半径为$b$、电流面密度为$-\sigma$的圆柱的组合。
且它们在自己轴线上产生的磁感应强度为0。那么由(1):
\begin{figure}
	\centering
	\includegraphics[width=0.9\textwidth]{./illus/2011_s1.ai}
	\caption{解答4 解析示意图}
\end{figure}
对于圆柱轴线,$a>b$时,如左图:
\[
B_1=\left|\dfrac{-\mu_0\sigma\cdot\pi b^2}{2\pi a}\right|=\dfrac{\mu_0Ib^2}{2\pi a(R^2-b^2)}
\]
$a<b$时,如右图:
\[
B_1=\left|\dfrac{-\mu_0\sigma\cdot\pi a^2}{2\pi a}\right|=\dfrac{\mu_0Ia}{2\pi(R^2-b^2)}
\]
对于空心圆柱轴线,
\[
B_2=\dfrac{\mu_0\sigma\cdot\pi a^2}{2\pi a}=\dfrac{\mu_0Ia}{2\pi(R^2-b^2)}
\]
综上,$\cdots$

\solves{5}
(1) 
\begin{align*}
\varepsilon_1&=\int_{a}^{b}(\vec{v}\times\vec{B})\cdot\di{l}\\
&=\int_{0}^{R}l\omega B\di{l}\\
&=\dfrac{1}{2}BR^2\omega\\
&=\pi NBR^2
\end{align*}
(2) (电动势方向是从轮子中心到轮子外侧,所以电流方向为)$b\rightarrow a$.

(3)
\begin{align*}
M&=\int_{a}^{b}\left|\vec{l}\times(I\di{\vec{l}}\times\vec{B})\right|\\
&=BI\int_{0}^{R}l\di{l}\\
&=\dfrac{1}{2}BIR^2
\end{align*}