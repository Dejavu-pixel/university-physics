%!TEX program = xelatex
\documentclass[blue]{./templete/qyxfnote}
\usepackage{xeCJK}
\usepackage{geometry}
\geometry{left=2.0cm,right=2.0cm,top=2.5cm,bottom=2.5cm}
\usepackage{amsmath}
\usepackage{amssymb}
\usepackage{graphics}
\setCJKmainfont{SimSun}
\newcommand{\di}[1]{\mathrm{d}#1}
\newcommand{\p}[2]{\frac{\partial #1}{\partial #2}}
\newcommand{\pp}[2]{\frac{\partial ^2 #1}{\partial #2 ^2}}
\newcommand{\dy}[2]{\frac{\di{#1}}{\di{#2}}}
\newcommand{\ddy}[2]{\frac{\mathrm{d} ^2 #1}{\mathrm{d} #2 ^2}}

\begin{document}
	\section*{第一章}
		21.
		\begin{gather*}
			\text{由图知:}\\
			\tan\alpha=\frac{|\vec{a_n}|}{|\vec{a_\tau}|}=\frac{\frac{v^2}{R}}{\dy{v}{t}}\\
			\therefore \dy{v}{t}\frac{1}{v^2}=\frac{1}{R\tan\alpha}\\
			\text{积分得:}-\frac{1}{v}=\frac{1}{R\tan\alpha}t+c\\
			\text{代入}t=0,v=v_0\\
			\therefore \frac{1}{v_0}-\frac{1}{v}=\frac{1}{R\tan\alpha}t
		\end{gather*}
		22.
		\begin{gather*}
			\vec{v}=\dy{s}{t}\vec{\tau}=(c+2dt)\vec{\tau}\\  
			\vec{a_n}=\frac{v^2}{R}\vec{n}=\frac{(c+2dt)^2}{R}\vec{n}\\
			\vec{a_\tau}=\dy{v}{t}\vec{\tau}=2d\tau\\
			\text{令}|\vec{a_n}|=|\vec{a_\tau}|\\
			\therefore \frac{(c+2dt)^2}{R}=2d\\
			\therefore t_1=\frac{\sqrt{2dR}-c}{2d}\left(t_2=\frac{-sqrt{2dR}-c}{2d}<0舍去\right)\\
			\text{要使t>0,条件为}\sqrt{2dR}-c>0,\text{即}2dR\leqslant c^2\\
			\therefore 
		\end{gather*}
\end{document}
		